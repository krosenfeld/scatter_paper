\documentclass[11pt,preprint]{aastex}
\usepackage{natbib}
\bibliographystyle{apj}
\usepackage{amsmath}
\providecommand{\e}[1]{\ensuremath{\times 10^{#1}}}

\newcommand{\sgra}{Sgr~A$^{\ast}$}
\defcitealias{jg15}{JG15}


\newcommand{\vdag}{(v)^\dagger}
\newcommand{\myemail}{}
\newcommand{\beq}[1]{\begin{equation}\label{#1}}
\newcommand{\eeq}{\end{equation}}
\newcommand{\unten}[1]{$_{\rm{#1}}$}
\newcommand{\tothe}[1]{$^{\rm{#1}}$}
\newcommand{\posion}[1]{$_{\rm{#1}}^{\rm{+}}$}
\DeclareGraphicsRule{.pdftex}{pdf}{*}{}

\shorttitle{Scattering}
\shortauthors{Rosenfeld}

\begin{document}

\title{TITLE}

\author{AUTHORS\altaffilmark{1}}
\altaffiltext{1}{Harvard-Smithsonian Center for Astrophysics, 60 Garden Street, Cambridge, MA 02138, USA}

\begin{abstract}

ABSTRACT

\end{abstract}

\keywords{KEYWORDS}

\section{Introduction} \label{intro}

Irregularities in the ionized interstellar medium (ISM) scatter radio waves, 
leading to familiar effects, such as scintillation of sources.  For 
sources that can be spatially resolved, the scattering also causes blurring and
distortion of the image, analogous to the deleterious effects of the atmosphere 
on optical observations.  Hence, as radio observations achieve ever higher 
angular resolution, these effects of the ISM on imaging become increasingly 
important considerations.  In this paper, we consider the effects of 
interstellar scattering on resolved images of the Galactic Center supermassive 
black hole, Sagittarius A$^\ast$ (\sgra).

\sgra\ has been studied with very-long-baseline interferometry (VLBI) for the 
last 40 years.  At wavelengths longer than a centimeter, the size of the image 
of \sgra\ grows with the squared observing wavelength.  This scaling is typical 
when the angular broadening of a source from ISM scattering is much larger than 
the angular size of the source.  At wavelengths shorter than a centimeter, the 
source angular size is larger than expected from this scaling, indicating the 
presence of intrinsic structure in addition to the angular broadening.  The 
shortest-wavelength VLBI observations to-date were performed at 1.3-mm with the 
Event Horizon Telescope \citep{doeleman08,fish11}.  At this wavelength, 
the angular broadening (${\sim}20~\mu{\rm as}$) is subdominant to intrinsic 
structure (${\sim}40~\mu{\rm as}$). 

Nevertheless, features such as a bright ring surrounding the ``shadow'' of the 
black hole can still be affected by the residual scatter-broadening.   Recently,
\citet{fish14} has shown that these effects can be effectively mitigated in 
the ensemble-average regime, where the ``blurring'' of scattering is 
deterministic.  Our goal is to use numerical simulations of the scattering to 
investigate the potential contribution of scattering-induced substructure for 
realistic images and observations.

\subsection{Interstellar Scattering}

The scattering of \sgra\ below about $2~{\rm THz}$ is ``strong'': the electric 
field seen by the observer consists of the superposition of many rays from 
different paths, whose phase from scattering differs by many radians across 
the extent of the scattered image.  For strong scattering, effects are divided 
into two distinct branches: diffractive and refractive. These branches have 
strikingly different properties.

Diffractive effects are narrowband, short-lived, and quenched by all but the 
most compact sources, such as pulsars.  Refractive effects are wideband, 
long-lived, and only begin to be quenched when a source exceeds the size of the 
scattered image of a point source (i.e., when the effects of scatter-broadening 
become subdominant to intrinsic structure).  Diffractive effects become stronger
at longer wavelengths, whereas refractive effects become stronger at shorter 
wavelengths.

Scattering also exhibits two opposite effects depending on the degree of 
averaging in time.  When averaged over long timescales (an ensemble-average), 
scattering acts to blur the image of a source.  In contrast, over shorter 
timescales, scattering introduces substructure in the image of the source, 
exaggerating gradients in the unscattered image.  Remarkably, the substructure 
can persist over long timescales, as was first shown by 
\citet{ng89} and \citet{gn89}, and can be an important consideration even when 
the scattering is subdominant to intrinsic structure, as was shown by 
\citet[][hereafter JG15]{jg15}. 
  
State conditions for validity of the geometric optics approximation (recast 
differently, with source coordinates rather than screen?):   

\begin{align}
\label{eq::Image_Approx}
\nonumber I_{\rm ss}(\textbf{x}) &\approx \int d^2\textbf{y}\, 
V_{\rm src}\left( (1+M) \textbf{y} \right)
e^{-i \textbf{y} \cdot \nabla \phi(\textbf{x}) }
e^{-\frac{i}{r_{\rm F}^{2}} \textbf{y}\cdot \textbf{x} }\\
&=I_{\rm src}\left( \textbf{x} + r_{\rm F}^2 \nabla \phi(\textbf{x}) \right).
\end{align}
Note that the action of scattering shuffles brightness elements of the source 
but does not conserve total flux.

{\bf need definitions} In this limit, the frequency evolution of the scattering is especially simple. The Fresnel scale, $r_{\rm F}$, is proportional to $\sqrt{\lambda}$. The behavior of $\phi(\textbf{x})$ follows from the refractive index of a dilute plasma at a frequency well above its plasma frequency, $\nu_{\rm p}$: $n \approx 1 - \frac{1}{2} \frac{\nu_{\rm p}^2}{\nu^2}$. Hence, $\phi(\textbf{x}) \propto (n-1)/\lambda \propto \lambda$. Thus, in this regime, scattering displaces a given brightness element of the source by an amount proportional to $\lambda^2$. 

Hence, we can rewrite the action of scattering as 
$I_{\rm ss}(\textbf{x}) \approx I_{\rm src}\left( \textbf{x} + \left( \frac{\lambda}{\lambda_0} \right)^2 \textbf{s}_0(\textbf{x}) \right)$.  Once the vector 
field $\textbf{s}_0(\textbf{x})$ is measured at a single wavelength $\lambda_0$ 
over some angular domain $D$, the scattering can be inverted at all frequencies 
over the same domain $D$.  If the intrinsic structure is sufficiently simple so 
that scattering features can be identified, then one can use the comparison of 
scattered images at different wavelengths to perform relative astrometry of 
source components (cite quasar stuff). 

A given solution at one observing wavelength can then be used to infer the 
scattering properties at other wavelengths.  And because of the steep scaling 
with wavelength, the scattering can be characterized at long wavelengths, where 
scatter-broadening is easily measured, and then applied to shorter wavelengths, 
where it is difficult to resolve or degenerate with structure from the 
unscattered source.

\section{Methods}

Should define scattered versus source image.  Mention code here.

\section{Scattering at different wavelengths}

Show realizations of scattering at 0.87, 1.3, and 3 mm.

\begin{figure}[t!]
\plotone{figs/sims.eps}
\figcaption{A source is shown in the left most panel and scattered instances of the
same source at 0.87, 1.3, and 3$\,$mm are displayed from left to right.  All images are smoothed by a 3$\,\mu$as Gaussian.
\label{fig:threelambda}}
\end{figure}

\section{Reconstruction of a scattered image}

Show how Vincent's method works on these images.  Does averaging really 
help for imaging with the bispectrum?  If not, can we estimate which baselines
are garbage for imaging?

\begin{figure}[t!]
\plotone{figs/imaging.eps}
\figcaption{These panels show images reconstructed using the bispectrum and the 
maximum entropy algorithm.  The original images are the same ones as 
shown in Figure \ref{fig:threelambda}.  The uv-samples were set by baselines 
between the
SMA, SMT, PV, ALMA, SPT, PDB, and LMT.  All images are smoothed by a Gaussian with FWHM of 10$\,\mu$as}
\end{figure}

\section{Ring Size}

One application of EHT observations is measuring properties of 
the photon ring.  This phenomenon traces where photons emitted from the 
accretion disk wrap multiple times around the black hole, creating a brightness 
concentration that outlines the black hole shadow.  The diameter of this ring
is $3\sqrt{3}GM/c^2$ which corresponds to an angular size of 52 $\mu$as for 
\sgra ($M_{\rm BH} = 4.3\times10^6\,M_\odot$ at a distance of $8.4\,$kpc).  The 
shape of this ring is a powerful test of general relativity 
with its shape and ellipticity indicating a possible deviation from the Kerr 
metric \citep{johannsen10,johannsen15}.  Consequently, it is important to properly estimate the 
uncertainties in the apparent ring shape due to distortion by scattering.

We consider the source to be a uniform annulus with width to diameter
ratio, $a = \Delta/d$, and the diameter, $d$, is the arithmetic mean of the 
inner and outer edge of the annulus.  For this simple source, the visibility 
amplitude is the different between two Bessel functions of the first kind:

\begin{equation}
|V_{\rm annulus}(\rho)| = \frac{ (1+a) J_1[\pi d (1+a) \rho]- (1-a)J_1[\pi d(1-a)\rho] 
}{\pi d \rho}.
\label{eq:annulus_vis}
\end{equation}

Equation \ref{eq:annulus_vis} shows that for a given width to diameter ratio 
$a$, the first (or any) null can be used to determine the diameter of the 
annulus, $d$.  The left panel in Figure \ref{fig:annulus_rms} shows the 
position of the first null of the visibility amplitude as a function of $a$.  
Using an annulus where $d = 57$ and $\Delta = 22$ (the fit from 
\citet[][]{doeleman08}), we generate 50 independent instances the scattered 
source and calculate a diameter using the
position of the first 
null along the major and minor axis of the scattering kernel.  The right panel
in Figure \ref{fig:annulus_rms} shows the resulting rms error for observations 
at 230 and 350$\,$GHz.  Even at 350$\,$GHz, the 1-$\sigma$ scatter is similar 
to the shift between a ring $(a=0)$ and full disk $(a=1)$ source model.  For
large enough ring sizes, the rms error appears to be independent of the ring
size.  However, there is a significant increase in the rms for annulli 
with diameters less than 40$\,\mu$as.  Here the scale of the scattering blur 
$22\,\mu$as) is as large as the radius, filling in the inner hole and 
introducing a bias towards smaller diameters.
For an annulus with $d=50\,\mu$as, the mean eccentricity that 
would be calculed using the major and minor axis diameters is 0.3 and 0.2 at 
230 and 350$\,$ GHz.

% Discuss eccentricity and asymmetry as defined by T. Johannsen?

% Discuss motivation for this model: simplicity, really.

% Discuss uncertainty in the kernel?  Maybe this should be its own section since
% it affects everything.

\begin{figure}[t!]
\plottwo{figs/annulus_null.eps}{figs/rms.eps}
\figcaption{{\it Left}: Position of the first null for a uniform annulus 
as a funtion of its diameter to width ratio.  The gray line indicates the 
annulus ratio, $a=22/57$, from the obsevations presented by 
\citet{doeleman08}. {\it Right}:  The root mean square error introduced by 
scattering on the derived annulus width.  Note that this effect is at least as 
large as the shift in the null going from a ring $(a=0)$ to a full disk $(a=1)$.
\label{fig:annulus_rms}}
\end{figure}

\section{Uncertainty in the scattering parameters}

\section{Conclusion} \label{conclusion}

\acknowledgments 

\clearpage

\bibliography{scattering}
\end{document}
